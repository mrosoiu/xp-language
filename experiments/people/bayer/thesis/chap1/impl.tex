% ********** Chapter 1 **********

\section{Implementation}
\label{sec:chap1:impl}

Um mit der Implementation beginnen zu k"onnen musste zun"achst ein als dynamisch ladbare Bibliothek verf"ugbares
PHP erzeugt werden. Hierzu wurde von \cite{PHPHP} ein PHP in der Version 5.2.0 im Quelltext heruntergeladen und 
"ubersetzt, nachdem es mittels des Kommandozeilenparameters "'--enable-embed=shared"' konfiguriert wurde. Die so
erzeugte \emph{libphp5.so} konnte, zusammen mit den vorhandenen Headerdateien, zur Entwicklung genutzt werden.

Nun musste noch daf"ur gesorgt werden, dass sich die vom JSR 223 verlangten Klassen aus \emph{javax.script} im
Classpath befinden. Nachdem allerdings die Klassen aus der Referenzimplementation aus den zum Teil in 
\ref{sec:chap1:prototype} beschriebenen Gr"unden nicht in Frage kamen, wurde von \cite{JAVAHP} ein
\emph{Java Development Toolkit (JDK)} der Version 6 heruntergeladen. Da dieses JDK nicht als System-VM genutzt
werden kann wurde ein Makefile erstellt, welches die zur "Ubersetzung und Ausf"uhrung n"otigen Kommandos
vereinfacht.

Nachdem diese Vorarbeiten geleistet waren konnte mit der eigentlichen Implementation begonnen werden.
Der erste Schritt war das Abbilden des ersten Use-Cases (verf"ugbare ScriptEngines auflisten), erstens um 
sicherzustellen dass die Entwicklungsumgebung den Anforderungen gerecht wird, und zweitens um einen ersten
Eindruck der JSR 223 API zu erhalten. Also wurde im Package \emph{samples} eine Klasse namens
\emph{EngineList} geschrieben, die einen ScriptEngineManager erzeugt, und mittels der Methode
\emph{getEngineFactories()} eine Liste aller verf"ugbaren ScriptEngines erstellt. Der Inhalt dieser
Liste wird dann mittels \emph{System.out.println()} ausgegeben. 
Nachdem dem Makefile ein Target namens "'list"' hinzugef"ugt wurde, ergab ein erstes Ausf"uhren dieser Klasse
zum einen keine Fehler, und zweitens dass dem JDK 6 mit \emph{Mozilla Rhino} bereits eine ScriptEngine beiliegt:
\begin{lstlisting}[caption=erste Tests]
# make list
found 1 available ScriptEngines:
Engine: Mozilla Rhino
#
\end{lstlisting}

Im Folgenden wurde die Klasse \emph{PHPScriptEngineFactory} erstellt, welche gem"a\ss \ref{sec:chap1:design:java} 
das Interface \emph{ScriptEngineFactory} implementiert, hierbei erw"ahnenswert sind die Methoden 
\emph{getProgram()} und \emph{getMethodCallSyntax()}, erstere erstellt aus als Strings vorliegenden
einzelnen Anweisungen ein ausf"uhrbares PHP-Programm, zweitere erzeugt aus den "ubergebenen Parametern
einen der PHP-Syntax entsprechenden Methodenaufruf. Wurde nun die EngineList ausgef"uhrt stellte sich heraus,
dass der ScriptEngineManager die neue ScriptEngine schon erkannte:
\begin{lstlisting}[caption=Neue ScriptEngine]
# make list
found 2 available ScriptEngines:
Engine: Mozilla Rhino
Engine: XP-Framework Turpitude PHP Engine
#
\end{lstlisting}

TODO: ScriptEngine angelegt, makefile total toll

Da\ss\ PHP urspr"unglich ausschlie\ss lich als CGI ausgef"uhrt wurde merkt man dem Kern der Sprache noch
deutlich an. So muss eine SAPI Funktionen aufrufen, welche den Beginn des sogenannten "'Requests"', respektive
dessen Ende anzeigen, auch m"ussen Funktionen zum setzen und lesen von HTTP-Headern und Cookies bereitgestellt
werden. Die n"otigen Aufrufe werden in der PHPScriptEngine von zwei privaten, nativen Methoden erledigt, welche
die ScriptEngine bei sich selbst aufruft.

TODO:
native Implementierung gestartet
HelloWorld geht
error\_cb  -> java exceptions
ScriptExecutor geschrieben
einfache Scripte mit Variablen scheinen zu gehen
phpinfo() segfaultet, muss weiter erforscht werden.
zval\_to\_jobject geschrieben, jni api suckt, zend api suckt haerter

zend\_eval\_string: echo ("'Hello World"'); tut, aber: wenn retval\_ptr == NULL dann wird das Script ausgef"uhrt,
wenn retval\_ptr valid zval* ist, dann wird die erste Zuweisung ausgef"uhrt, und der R"uckgabewert == FAILURE;
TestScript:
\begin{lstlisting}[caption=Testscript f"ur zend\_eval\_string()]
$bla = "Test";
for ($i=0; $i<10; $i++) {
    printf("Hello %s\n", $bla);
}
\end{lstlisting}

zend\_eval\_string nicht der weg - compile\_string. Deswegen: gleich mal Compilable implementiert, PHPCompiledScript
angelegt


ini file -> wie?
Segfaults: --enable-debug, C-Frontend schreiben





% ********** End of chapter **********
