% ********** Appendix 1 **********
\chapter{Turpitude - Programmierbeispiele}
\label{sec:app1}

Turpitude ist eine JSR223-Implementation \cite{JSRHP}, die es einem Anwender erlaubt PHP-Skripte
aus Java heraus auszuf"uhren, Funktionen und Objektmethoden innerhalb solcher Skripte aufzurufen
und R"uckgabewerte dieser Skripte, Funktions- und Methodenaufrufe zur"uckzuerhalten.
Ausserdem erweitert Turpitude den PHP-Interpreter um die M"oglichkeit aus PHP heraus
Java-Objekte zu erzeugen und Java-Methoden auf diesen Objekten aufzurufen.
An dieser Stelle soll eine kurze Einf"uhrung in die notwendigen Vorbereitungen, sowie
in die API sowohl in Java als auch in PHP erfolgen.
Diese Dokumentation richtet sich vor allem an Entwickler die planen mit Turpitude zu arbeiten.

% ********** Appendix 1 **********
\section{Installation}
\label{sec:app1:inst}

Die Installationsanweisungen in diesem Kapitel beziehen sich im Allgemeinen auf ein unixoides
System, sind aber leicht f"ur Windows anpassbar.

\subsection{Java}
\label{sec:app1:inst:java}

Zun"achst muss ein Java-SDK der Version 6.0 oder h"oher vorhanden sein, zu finden auf der
Java-Homepage \cite{JAVAHP}. Das Installationsverzeichnis des SDK wird im weiteren Verlauf als
\emph{JAVA\_HOME} bezeichnet.

\subsection{PHP}
\label{sec:app1:inst:php}
Weiterhin muss PHP aus den Quellen "ubersetzt werden. Hierzu muss zun"achst der PHP-Sourcecode in
einer Version 5.2.0 oder h"oher von der PHP-Homepage \cite{PHPHP} oder direkt aus dem PHP 
CVS-Repository heruntergeladen werden. Das Verzeichnis der PHP-Quellen wird im weiteren Verlauf
als \emph{PHP\_HOME} bezeichnet. PHP kann nun mittels des Befehls \emph{configure} konfiguriert werden,
der Anwender kann hier jeden gewohnten Parameter benutzen, wichtig ist nur dass der Paramter
\emph{enable-embed} auf \emph{shared} gesetzt wird. Nach dieser Konfiguration kann PHP mittels des
Befehls \emph{make} "ubersetzt werden:

\begin{lstlisting}[caption=Konfigurieren und "Ubersetzen von PHP]
~ # cd $PHP_HOME
php-5.2.0 # ./configure --enable-embed=shared 
php-5.2.0 # ./make
\end{lstlisting}

Im Verzeichnis \emph{PHP\_HOME/libs} sollte sich nun eine Bibliothek names \emph{libphp5.so} befinden.

\subsection{Turpitude}
\label{sec:app1:inst:turp}

Wechseln Sie nun in das Turpitude-Verzeichnis und "offnen Sie die Datei \emph{Makefile} mit einem 
Texteditor ihrer Wahl. "Andern sie die gekennzeichneten Zeilen ihrem System entsprechend:

\begin{lstlisting}[caption=Anpassen des Makefiles]
# vim Makefile
JAVA_HOME = /home/nsn/jdk1.6.0
PHP_HOME = /home/nsn/devel/php/php-5.2.0
\end{lstlisting}

Ein Aufruf des Befehls \emph{make} erzeugt nun zwei Dateien: \emph{turpitude.jar}, welches die
ben"otigten Java-Klassen enth"alt, sowie ein \emph{libturpitude.so}.

% ********** End of appendix **********

% ********** Appendix 1 **********
\section{Ausf"uhren}
\label{sec:app1:exec}

Um nun die beigef"ugten Beispiele auszuf"uhren, oder um eigene Applikationen entwickeln zu 
k"onnen muss die Java-Laufzeitumgebung entsprechend angepasst werden:
stellen Sie zun"achst sicher dass sich die in \ref{sec:app1:inst:php} erzeugte libphp5.so in
einem Verzeichnis befindet, aus dem das Betriebssystem Programmbibliotheken laden kann,
meist \emph{\$LD\_LIBRARY\_PATH}. Dies geschieht entweder durch ein installieren der libphp5.so
in den Systembibliothekspfad, oder in dem die entsprechende Variable umgesetzt wird:

\begin{lstlisting}[caption=Anpassen des Makefiles]
# export LD_LIBRARY_PATH=.
\end{lstlisting}

Weiterhin muss die in \ref{sec:app1:inst:turp} erzeugte libturpitude.so in dem Verzeichnis liegen,
aus dem Java Programmbibliotheken l"adt. Dieses Verzeichnis steht im Systemproperty \emph{java.library.path}.
Auch hier kann entweder die Bibliothek in das entsprechende Verzeichnis kopiert werden, oder 
der JVM kann beim Start ein alternativer Pfad "ubergeben werden:

\begin{lstlisting}[caption=Ausf"uhren eines Turpitude-Programmes]
# java -cp ./turpitude.jar \
  -Djava.library.path=. \
  net.xp_framework.turpitude.samples.EngineList
\end{lstlisting}

Dieser Befehl sollte nun alle verf"ugbaren ScriptEngines auflisten, Turpitude sollte hier erscheinen.


% ********** End of appendix **********

\section{Beispiele}
% ********** Appendix 1 **********
\subsection{Hello World}
\label{sec:app1:hello}

Als erstes soll nun ein einfaches Programm entstehen, das mittels PHP die
Zeichenfolge "'Hello World!"' auf dem Bildschirm ausgibt. Dazu muss zun"achst
einmal eine Instanz der PHP-Scriptengine erzeugt werden:
\begin{lstlisting}[caption=Erzeugen der PHP-ScriptEngine]
ScriptEngineManager mgr = new ScriptEngineManager();
ScriptEngine eng = mgr.getEngineByName("turpitude");
\end{lstlisting}
Der ScriptEngineManager stellt Methoden zum Auffinden von ScriptEngines zur Verf"ugung,
die Methode \emph{getEngineByName()} gibt uns die gew"unschte Instanz zur"uck. Nun
kann das Script ausgef"uhrt werden:
\begin{lstlisting}[caption=Hello World Skript]
try {
    eng.eval("echo(\"Hello World!\n\");");
} catch(ScriptException e) {
    System.out.println("ScriptException caught:");
    e.printStackTrace();
}
\end{lstlisting}
Die Methode \emph{eval()} der ScriptEngine f"uhrt ein in einem String gespeichertes
Skript aus. Da es sich hierbei um Java-Quelltext handelt ist es n"otig etwaige Sonderzeichen
mit einem "'$backslash$"' zu maskieren.

Der Quelltext dieses Programmes befindet sich in der Datei \emph{HelloWorld.java}, und kann
mittels des Befehls \emph{make hello} ausgef"uhrt werden.

% ********** End of appendix **********

% ********** Appendix 1 **********
\subsection{Skriptdateien ausf"uhren}
\label{sec:app1:execute}

Wie sich in \ref{sec:app1:hello} gezeigt hat erweist es sich als umst"andlich PHP-Quelltext
in einem Java-Programm einzubetten. Alternativ kann das Skript sich auch in einer Datei befinden,
was den zus"atzlichen Vorteil bringt nicht jedes Mal die Java-Klasse neu "ubersetzen zu m"ussen 
wenn sich das Skript "aendert.

Hierzu muss ein \emph{java.io.Reader} erzeugt werden, aus welchem das Skript geladen werden kann:
\begin{lstlisting}[caption=Laden eines Skriptes aus einer Datei]
FileReader r = new FileReader(filename);
\end{lstlisting}
Diesen Reader kann man nun der ScriptEngine "ubergeben:
\begin{lstlisting}[caption="Ubergabe des Readers]
Object retval = null;
try {
    retval = eng.eval(r);
} catch(PHPCompileException e) {
    System.out.println("Compile Error:");
    e.printStackTrace();
} catch(PHPEvalException e) {
    System.out.println("Eval Error:");
    e.printStackTrace();
} catch(ScriptException e) {
    System.out.println("ScriptException caught:");
    e.printStackTrace();
}
\end{lstlisting}

Hier zeigen sich gleich zwei Besonderheiten die es bei der Benutzung von Turpitude zu beachten
gilt: Neben der vom JSR223-Standard beschriebenen \emph{ScriptException} wirft Turpitude noch 
\emph{PHPCompileException}s und \emph{PHPEvalException}s, was es dem Anwender erlaubt 
zwischen "Ubersetzungs- und Laufzeitfehlern zu unterscheiden.
Ausserdem k"onnen PHP-Skripte einfach mittels "'return"' Werte zur"uckgeben, welche dann in Java
entweder als skalare Typen, oder aber als Instanzen der Klasse \emph{PHPObject} repr"asentiert
werden. Ein PHPObject enth"alt eine \emph{java.util.HashMap} die alle Attribute des PHP-Objektes
unter dem jeweiligen Namen gespeichert hat.

Der Quelltext dieses Programmes befindet sich in der Datei \emph{ScriptExec.java}, und kann
mittels des Befehls \emph{make exec} ausgef"uhrt werden.

% ********** End of appendix **********

% ********** Appendix 1 **********
\subsection{Skripte "Ubersetzen}
\label{sec:app1:compile}

Anstatt Skripte st"andig zu laden und auszuf"uhren bietet Turpitude auch die M"oglichkeit
ein Skript einmal zu "ubersetzen und dann mehrfach auszuf"uhren. Hierzu implementiert
die ScriptEngine das JSR223-Interface \emph{javax.script.Compilable}. Um ein Skript
zu "ubersetzen muss statt \emph{eval()} die Methode \emph{compile()} aufgerufen werden:
\begin{lstlisting}[caption="Ubersetzen eine Skriptes]
Compilable comp = (Compilable)eng;
CompiledScript script = comp.compile(src);
\end{lstlisting}
Ein derart "ubersetztes Skript l"a\ss t sich beliebig oft ausf"uhren:
\begin{lstlisting}[caption=Ausf"uhren des "ubersetzten Skriptes]]
for (int i=0; i<5; i++) {
    System.out.println("executing " + i);
    script.eval();
}
\end{lstlisting}
Auch bei "ubersetzten Skripten gelten die Regeln Besonderheiten aus den vorhergehenden
Beispielen.

Der Quelltext dieses Programmes befindet sich in der Datei \emph{CompileSample.java}, 
und kann mittels des Befehls \emph{make compiledscript} ausgef"uhrt werden.

% ********** End of appendix **********

% ********** Appendix 1 **********
\subsection{Java-Objekte in PHP}
\label{sec:app1:jobj}

Turpitude erlaubt es dem Anwender aus PHP heraus Java-Objekte zu erzeugen und Methoden
auf diesen aufzurufen, allerdings gilt es dabei einige Besonderheiten zu beachten.

Zun"achst muss eine Instanz der Klasse \emph{TurpitudeEnvironment} gefunden werden.
Dies kann entweder mittels eines einfach "'new"' geschehen, allerdings findet sich
eine solche auch im PHP-Superglobal \emph{\$\_SERVER}:
\begin{lstlisting}[caption=TurpitudeEnvironment-Instanz]
$turpenv = $_SERVER["TURP_ENV"];
\end{lstlisting}
Der Name, unter dem sich die Instanz finden l"a\ss t, ist in der PHPScriptEngine konfigurierbar,
der Default lautet "'TURP\_ENV"'.
Nachdem dieser Schritt getan ist kann eine Instanz der Klasse \emph{TurpitudeJavaClass}, 
welche eine Java-Klasse repr"asentiert, erzeugt werden:
\begin{lstlisting}[caption=Java-Klassen finden]
$class = $turpenv->findClass("net/xp_framework/turpitude/samples/ExampleClass");
\end{lstlisting}
Wichtig ist hierbei das Format des Klassennamens - er entspricht der JNI-Syntax \cite{JNIHP}.
Mit diesem Objekt k"onnen nun bereits statische Methoden aufgerufen werden, allerdings nicht
ohne zuvor eine Instanz der Klasse \emph{TurpitudeJavaMethod} erzeugt zu haben, diese Klasse
beschreibt eine Java-Methode. Um statisch aufrufbare Methoden zu finden muss die Funktion
\emph{findStaticMethod()} genutzt werden, welche zwei Parameter erwartet: den Methodennamen
und die Methodensignatur, wieder in JNI-Syntax kodiert:
\begin{lstlisting}[caption=statische Methoden finden]
$method = $class->findStaticMethod('staticMethod', '(I)Ljava/lang/String;');
\end{lstlisting}
Jetzt kann die statische Methode aufgerufen werden:
\begin{lstlisting}[caption=statischer Methodenaufruf]
$retval = $class->invokeStatic($method, 17);
\end{lstlisting}
Der erste Parameter der Methode \emph{invokeStatic()} ist die aufzurufende Methode,
alle weiteren werden der Java Virtual Machine als Methodenparameter "ubergeben, es
obliegt also dem Anwender sicherzustellen dass die richtige Anzahl Parameter, als auch die
richtigen Typen "ubergeben werden.
Um nun eine Instanz der Java-Klasse erzeugen zu k"onnen muss zun"achst der Konstruktor
gefunden werden, un zwar mittels der Methode \emph{findMethod()}, die wieder die Signatur
des gew"unschten Konstruktors als Parameter erwartet:
\begin{lstlisting}[caption=Konstruktoren finden]
$constructor = $class->findConstructor('(ILjava/lang/String;)V');
\end{lstlisting}
Der so gefunden Konstruktor kann nun als erster Parameter f"ur die Methode \emph{create()}
verwandt werden, jeder weitere Parameter wird wieder an die JVM weitergeleitet. Auf diese
Weise erh"alt man ein Objekt der Klasse \emph{TurpitudeJavaObject}:
\begin{lstlisting}[caption=Konstruktoren finden]
$instance = $class->create($constructor, 1337, 'eleet');
\end{lstlisting}
Um nun Attribute dieser Instanz auslesen und setzen zu k"onnen stehen die beiden Methoden
\emph{javaGet()} und \emph{javaSet()} zur Verf"ugung. Erstere erwartet zwei Parameter, den
Namen des zu setzenden Attributes und dessen Signatur, wiederum JNI-kodiert. Letztere erwartet
einen dritten Parameter, den neuen Wert:
\begin{lstlisting}[caption=Attribute lesen und schreiben]
$int = $instance->javaGet('intval', 'I');
$instance->javaSet('intval', 'I', 666);
\end{lstlisting}
Man beachte allerdings: auf diese Weise lassen sich nicht nur "offentliche Attribute lesen
und schreiben, sondern auch private. Auch hier obliegt dem Anwender eine gewisse Sorgfaltspflicht.
Will der Anwender nur auf "offentliche Attribute zugreiffen kann er dies wie gewohnt bei ganz
normalen PHP-Objekten tun:
\begin{lstlisting}[caption=Attribute lesen und schreiben]
$int = $instance->intval;
$instance->intval = 666;
\end{lstlisting}
Weiterhin lassen sich nat"urlich auch Methoden des Java-Objektes aufrufen, allerdings nicht ohne
vorher wieder ein TurpitudeJavaMethod-Objekt zu erzeugen, und zwar unter Verwendung der
Methode \emph{findMethod()}, welche analog zu \emph{findStaticMethod()} funktioniert. Die derart
erzeugte Methode wird - "ahnlich wie bei statischen Methoden - als erster Parameter f"ur die
Methode \emph{javaInvoke()} verwandt:
\begin{lstlisting}[caption=Methoden aufrufen]
$method = $class->findMethod('setValues', '(ILjava/lang/String;)V');
$instance->javaInvoke($method, 1338, 'eleeter');
\end{lstlisting}
Allerdings k"onnen Java-Methoden auch mit etwas weniger Aufwand aufgerufen werden:
jeglicher Methodenaufruf auf einem Objekt der Klasse \emph{TurpitudeJavaObject} wird an die
JVM weitergeleitet, wenn es sich nicht um einen Aufruf mit dem Namen \emph{javaGet}, \emph{javaSet}
oder \emph{javaInvoke} handelt.
\begin{lstlisting}[caption=direkter Methodenaufruf]
$result = $instance->getDate();
\end{lstlisting}
Das \emph{TurpitudeEnvironment} bietet noch eine weitere n"utzliche Methode an: \emph{instanceOf}.
Sie funktioniert "ahnlich wie der Java-Operator \emph{instanceof}, allerdings nur
auf Objekten der Klasse \emph{TurpitudeJavaObject}. Diese Methode erwartet als ersten Parameter
das zu "uberpr"ufende Objekt, und als zweiten Parameter entweder einen JNI-kodierten Klassennamen,
oder eine Instanz der Klasse \emph{TurpitudeJavaClass}:
\begin{lstlisting}[caption=instanceOf]
$turpenv->instanceOf($instance, 'java/util/Date');
$turpenv->instanceOf($instance, $class);
\end{lstlisting}
Schlussendlich soll noch erw"ahnt werden, dass PHP-Objekte der Klassen
\emph{TurpitudeJavaClass} und \emph{TurpitudeJavaObject} nicht als PHPObject, sondern als "'echte"'
Java-Objekte an die JVM zur"uckgegeben werden.

Der Quelltext dieses Beispieles befindet sich in der Datei \emph{ObjectSample.java}, 
und kann mittels des Befehls \emph{make objects} ausgef"uhrt werden. Der Quelltext der
verwendeten Beispielklasse befindet sich in der Datei \emph{ExampleClass.java}.

% ********** End of appendix **********

% ********** Appendix 1 **********
\subsection{Java-Arrays in PHP}
\label{sec:app1:arrays}

Einen Sonderfall stellen Java-Arrays dar. Sie werden in PHP nicht durch ein \emph{TurpitudeJavaObject},
sondern durch eine weitere Klasse, das \emph{TurpitudeJavaArray} repr"asentiert. Ein solches Objekt bietet
haupts"achlich drei Methoden: \emph{getLength()} gibt die L"ange des Java-Arrays zur"uck, mit \emph{get()} 
l"asst sich ein Element des Arrays auslesen, und \emph{set()} erm"oglicht das Setzen eines Elementes:
\begin{lstlisting}[caption=Normaler Zugriff auf ein TurpitudeJavaArray]
$method = $class->findMethod(
    'getStringArray', 
    '()[Ljava/lang/String;');
$array = $instance->javaInvoke($method);
$length = $array->getLength();
$val = $array->get(0);
$array->set(0, 'test');
\end{lstlisting}
Zus"atzlich implementiert das \emph{TurpitudeJavaArray} das PHP-Interface \emph{ArrayAccess}, und erlaubt
dem Anwender somit den direkten Zugriff auf Elemente mittels des []-Operators:
\begin{lstlisting}[caption=Klammern-Operator]
$val = $array[0];
$array[0] = 'test';
\end{lstlisting}
Weiterhin implementiert die Klasse das Interface, \emph{IteratorAggregate}, und bietet folglich die Methode
\emph{getIterator()} an, die eine Instanz der Klasse \emph{TurpitudeJavaArrayIterator} zur"uckgibt, welches
wiederum das Interface \emph{Iterator} implementiert. Somit ist es m"oglich auf gewohnte Weise "uber
Java-Arrays zu iterieren:
\begin{lstlisting}[caption=Iterator]
$iterator = $array->getIterator();
while ($iterator->valid()) {
    $row = $iterator->current();
    $key = $iterator->key();
    var_dump($row);
    var_dump($key);
    $iterator->next();
}
\end{lstlisting}
Die Methoden des \emph{TurpitudeJavaArrayIterator} k"onnen allerdings nicht nur direkt aufgerufen werden,
wie Klassen die das Interface \emph{Iterator} implementieren kann auch er in einer \emph{foreach}-Schleife
verwendet werden:
\begin{lstlisting}[caption=Iterator in einer foreach-Schleife]
foreach($iterator as $key => $row) {
  var_dump($key);
  var_dump($row);
}
\end{lstlisting}
Weiterhin ist es m"oglich Java-Arrays direkt zu erzeugen, dazu wird die Methode \emph{newArray()} beim
\emph{TurpitudeEnvironment} aufgerufen, die zwei Argumente erwartet, den JNI-kodierten Typen
und die gew"unschte Gr"o\ss e des Arrays:
\begin{lstlisting}[caption=Erzeugen von Arrays]
$arr = $turpenv->newArray('I', 5);
$arr2 = $turpenv->newArray('Ljava/lang/Object;');
\end{lstlisting}

Der Quelltext dieses Beispieles befindet sich in der Datei \emph{ArraySample.java}, 
und kann mittels des Befehls \emph{make arrays} ausgef"uhrt werden. Der Quelltext der
verwendeten Beispielklasse befindet sich in der Datei \emph{ExampleClass.java}.

% ********** End of appendix **********

% ********** Appendix 1 **********
\subsection{Java-Exceptions in PHP}
\label{sec:app1:exceptions}

Ein weiteres wichtiges Merkmal der Programmiersprache Java sind Exceptions. 
Turpitude bietet eine Reihe von Methoden die das Werfen und Fangen von Java-Exceptions
in PHP erlauben.

Zun"achst bringt das \emph{TurpitudeEnvironment} zwei Methoden mit die dem Anwender
das Werfen von Java-Exceptions erlauben: \emph{throw()} und \emph{throwNew()}. \emph{throw()}
erwartet als einzigen Parameter eine Instanz einer Java-Exception und wirft diese direkt,
w"ahrend \emph{throwNew()} zwei Parameter erwartet - den JNI-kodierten Klassennamen der
zu werfenden Exception, sowie die Nachricht (message) die diese Exception enthalten soll.
\begin{lstlisting}[caption=Werfen von Exceptions]
$class = $turpenv->findClass('java/lang/Exception');
$constructor = $class->findConstructor('(Ljava/lang/String;)V');
$instance = $class->create($constructor, 'Test');
$turpenv->throw($instance);
...
$turpenv->throwNew('java/lang/IllegalArgumentException', 'Test');
\end{lstlisting}
Um zu "uberpr"ufen, ob eine Exception aufgetreten ist kann die Methode \emph{exceptionOccurred()}
genutzt werden, welche die gerade aktuelle Exception zur"uckgibt:
\begin{lstlisting}[caption=Exceptions aufgetreten?]
if ($exc = $turpenv->exceptionOccurred()) {
    ...
}
\end{lstlisting}
Hat der Anwender die Fehlerbehandlung abgeschlossen muss die Methode \emph{exceptionClear()} aufgerufen
werden, um der JVM mitzuteilen dass sie die Exception als gefangen betrachten soll. Tut der Anwender
dies nicht gilt die Exception als nicht behandelt, und liegt weiterhin an:
\begin{lstlisting}[caption=Exceptions behandeln]
$turpenv->throwNew('java/lang/IllegalArgumentException', 'Test');
if ($exc = $turpenv->exceptionOccurred()) {
    printf(\"Msg: %s\\n\", $exc->toString('()Ljava/lang/String;'));
    $turpenv->exceptionClear();
}
\end{lstlisting}

Der Quelltext dieses Beispieles befindet sich in der Datei \emph{ExceptionSample.java}, 
und kann mittels des Befehls \emph{make exceptions} ausgef"uhrt werden.

% ********** End of appendix **********

% ********** Appendix 1 **********
\subsection{Der ScriptContext}
\label{sec:app1:context}

Die JSR223-Spezifikation beschreibt den bevorzugten Weg Daten an das ausgef"uhrte Skript zu
"ubergeben, und Daten aus dem Skript an die Java-Applikation zur"uckzu"ubermitteln: Den
ScriptContext. Dieser nimmt Objekte auf und ist im Skript verf"ugbar. Bei Turpitude sind diese
Objekte allerdings nicht nur Kopien der Java-Objekte, sondern Referenzen. "Anderungen an diesen
Objekten haben direkte Auswirkungen auf die Java-Applikation.

Zun"achst muss dem ScriptContext allerdings ein Objekt zugef"uhrt werden, dies geschieht 
- nachdem die Referenz des ScriptContext von der ScriptEngine geholt wurde - mittels
der Methode \emph{setAttribute()}. Der dritte Parameter beschreibt den sogenannten \emph{Scope},
den G"ultigkeitsbereich, des Wertes.
\begin{lstlisting}[caption=Kontext bef"ullen]
ScriptContext ctx = eng.getContext();
StringBuffer sb = new StringBuffer();
sb.append("before Script\n");
ctx.setAttribute("buffer", sb, ScriptContext.ENGINE_SCOPE);
\end{lstlisting}
Weitere Methoden Objekte im Kontext abzulegen sowie weitere Details zum Scope entnehmen Sie bitte 
der Java API-Dokumentation der Klasse \emph{javax.script.ScriptContext}.
Nun kann das ausgef"uhrte PHP-Skript auf den StringBuffer im ScriptContext zugreiffen:
\begin{lstlisting}[caption=Kontext in PHP]
$ctx = $turpenv->getScriptContext();
$buffer = $ctx->getAttribute(
                    '(Ljava/lang/String;)Ljava/lang/Object;', 
                    'buffer'
                );
$buffer->append(
            '(Ljava/lang/String;)Ljava/lang/StringBuffer;', 
            'Script Value'
        );
\end{lstlisting}
Liest man nun nachdem das Skript beendet ist den StringBuffer aus, enth"alt er nicht nur die
Zeile "'before Script"', sondern zus"atzlich die Zeile "'Script Value"'.

Der Quelltext dieses Beispieles befindet sich in der Datei \emph{ContextSample.java}, 
und kann mittels des Befehls \emph{make context} ausgef"uhrt werden.

% ********** End of appendix **********

% ********** Appendix 1 **********
\subsection{Aufrufen von PHP-Methoden}
\label{sec:app1:invoke}

Bisher wurde haupts"achlich besprochen wie ein PHP-Skript auf Java-Klassen, deren Attribute und
Methoden zugreiffen kann. In diesem Abschnitt soll der umgekehrte Weg gegangen, und aus Java heraus 
Methoden und Top-Level Funktionen eines PHP-Skriptes aufgerufen werden. Hierzu wird zun"chst
ein solches Skript geschrieben, in dem eine Klasse definiert wird, und das eine Funktion enth"alt die 
eine Instanz dieser Klasse zur"uckgibt:
\begin{lstlisting}[caption=PHP-Skript]
function useless($i) {
    return new foo($i);
}
 
class foo {
  var $val = '';
  function __construct($i) {
    $this->val = $i;
  }
  function bar($i) {
    return 'foo::bar::'.$i.' ('.$this->val.')';
  }
}
\end{lstlisting}
Um nun solche Skript-Funktionen aufzurufen definiert der JSR223 ein weiteres Interface, \emph{Invocable}.
Laut Spezifikation soll eigentlich lediglich die jeweilige \emph{ScriptEngine} dieses Interface implementieren,
allerdings w"are dann nicht klar auf welchem Skript die Funktion aufgerufen werden soll. Deswegen implementiert
das \emph{PHPCompiledScript} dieses Interface ebenfalls, und die \emph{PHPScriptEngine} leitet die Interface-Aufrufe
an das zuletzt "ubersetzte Skript weiter. Zun"achst muss das Skript also "ubersetzt werden, siehe hierzu auch
Kapitel \ref{sec:app1:invoke}.
\begin{lstlisting}[caption="Ubersetzen]
ScriptEngine eng = mgr.getEngineByName("turpitude");
Compilable comp = (Compilable)eng;
CompiledScript script = comp.compile(/*Source*/);
\end{lstlisting}
Nachdem das Skript "ubersetzt ist kann die globale Funktion aufgerufen werden:
\begin{lstlisting}[caption=Funktionsaufruf]
Invocable inv = (Invocable)script;
Object phpobj = inv.invokeFunction("useless", "Function Value");
\end{lstlisting}
Das phpobj h"alt nun eine Referenz auf die Instanz der PHP-Klasse \emph{foo}. Jetzt kann bei dieser
die Methode \emph{bar()} aufgerufen werden:
\begin{lstlisting}[caption=Methodenaufruf]
Object retval = inv.invokeMethod(phpobj, "bar", "Method Value");
\end{lstlisting}
Die Methoden \emph{invokeFunction()} und \emph{invokeMethod()} haben sehr "ahnliche Signaturen, beide
erwarten den Namen der Funktion/Methode als String und die zu "ubergebenden Argumente als weitere Parameter,
\emph{invokeMethod()} erwartet zus"atzlich ein PHPObjekt auf dem die Methode aufgerufen werden soll.

Ein weiteres Feature des \emph{Invocable}-Interfaces ist die Methode \emph{getInterface()}, die in zwei
Ausf"uhrungen existiert. Diese Methode gibt eine in der jeweiligen Skriptsprache implementierte Instanz eines
"ubergebenen Interfaces zur"uck, auf die dann in Java wie auf ein "'echtes"' Java-Objekt zugegriffen werden kann.
Zur Demonstation ein simples Java-Interface:
\begin{lstlisting}[caption=Java-Interface]
public interface ExampleInterface {
    public String bar(String s);
}
\end{lstlisting}
Zuf"alligerweise implementiert die PHP-Klasse \emph{foo} bereits dieses Interface, und da wir dies
wissen, k"onnen wir das phpobj \emph{getInterface()} "ubergeben, und diese Methode gibt uns ein 
entsprechendes Java-Objekt zur"uck, auf dem wir ganz normal arbeiten k"onnen:
\begin{lstlisting}[caption=getInterface() mit "ubergebenem Objekt]
ExampleInterface exint;
exint = inv.getInterface(phpobj, ExampleInterface.class);
exint.bar("interface call");
\end{lstlisting}
Falls kein passendes Objekt zur Verf"ugung steht kann der Anwender die Methode \emph{getInterface()}
auch ohne den ersten Parameter aufrufen, dann versucht Turpitude anhand des simplen Klassennamens
(bei \emph{java.lang.String} w"are dieser \emph{String}) eine passende Klasse zu finden, und mittels
des Default-Konstruktors eine Instanz dieser Klasse zu erzeugen. Erweitern wir als das obige
PHP-Skript um folgende Zeilen:
\begin{lstlisting}[caption=Interface Implementation in PHP]
class ExampleInterface {
  function bar($i) {
    return 'ExampleInterface::bar::'.$i;
  }
}
\end{lstlisting}
K"onnen wir einfach folgenden Java-Code benutzen um die Methode \emph{bar()} aufzurufen,
diesmal bei der PHP-Klasse \emph{ExampleInterface}:
\begin{lstlisting}[caption=getInterface() mit "ubergebenes Objekt]
exint = inv.getInterface(ExampleInterface.class);
exint.bar("interface call");
\end{lstlisting}


Der Quelltext dieses Beispieles befindet sich in der Datei \emph{InvocableSample.java}, 
und kann mittels des Befehls \emph{make objects} ausgef"uhrt werden. Der Quelltext des
verwendeten Beispielinterfaces befindet sich in der Datei \emph{ExampleInterface.java}.

% ********** End of appendix **********


% ********** End of appendix **********
