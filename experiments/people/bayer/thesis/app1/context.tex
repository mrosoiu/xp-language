% ********** Appendix 1 **********
\subsection{Der ScriptContext}
\label{sec:app1:context}

Die JSR223-Spezifikation beschreibt den bevorzugten Weg Daten an das ausgef"uhrte Skript zu
"ubergeben, und Daten aus dem Skript an die Java-Applikation zur"uckzu"ubermitteln: Den
ScriptContext. Dieser nimmt Objekte auf und ist im Skript verf"ugbar. Bei Turpitude sind diese
Objekte allerdings nicht nur Kopien der Java-Objekte, sondern Referenzen. "Anderungen an diesen
Objekten haben direkte Auswirkungen auf die Java-Applikation.

Zun"achst muss dem ScriptContext allerdings ein Objekt zugef"uhrt werden, dies geschieht 
- nachdem die Referenz des ScriptContext von der ScriptEngine geholt wurde - mittels
der Methode \emph{setAttribute()}. Der dritte Parameter beschreibt den sogenannten \emph{Scope},
den G"ultigkeitsbereich, des Wertes.
\begin{lstlisting}[caption=Kontext bef"ullen]
ScriptContext ctx = eng.getContext();
StringBuffer sb = new StringBuffer();
sb.append("before Script\n");
ctx.setAttribute("buffer", sb, ScriptContext.ENGINE_SCOPE);
\end{lstlisting}
Weitere Methoden Objekte im Kontext abzulegen sowie weitere Details zum Scope entnehmen Sie bitte 
der Java API-Dokumentation der Klasse \emph{javax.script.ScriptContext}.
Nun kann das ausgef"uhrte PHP-Skript auf den StringBuffer im ScriptContext zugreiffen:
\begin{lstlisting}[caption=Kontext in PHP]
$ctx = $turpenv->getScriptContext();
$buffer = $ctx->getAttribute(
                    '(Ljava/lang/String;)Ljava/lang/Object;', 
                    'buffer'
                );
$buffer->append(
            '(Ljava/lang/String;)Ljava/lang/StringBuffer;', 
            'Script Value'
        );
\end{lstlisting}
Liest man nun nachdem das Skript beendet ist den StringBuffer aus, enth"alt er nicht nur die
Zeile "'before Script"', sondern zus"atzlich die Zeile "'Script Value"'.

Der Quelltext dieses Beispieles befindet sich in der Datei \emph{ContextSample.java}, 
und kann mittels des Befehls \emph{make context} ausgef"uhrt werden.

% ********** End of appendix **********
