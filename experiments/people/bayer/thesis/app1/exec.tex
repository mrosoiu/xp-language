% ********** Appendix 1 **********
\section{Ausf"uhren}
\label{sec:app1:exec}

Um nun die beigef"ugten Beispiele auszuf"uhren, oder um eigene Applikationen entwickeln zu 
k"onnen muss die Java-Laufzeitumgebung entsprechend angepasst werden:
stellen Sie zun"achst sicher dass sich die in \ref{sec:app1:inst:php} erzeugte libphp5.so in
einem Verzeichnis befindet, aus dem das Betriebssystem Programmbibliotheken laden kann,
meist \emph{\$LD\_LIBRARY\_PATH}. Dies geschieht entweder durch ein installieren der libphp5.so
in den Systembibliothekspfad, oder in dem die entsprechende Variable umgesetzt wird:

\begin{lstlisting}[caption=Anpassen des Makefiles]
# export LD_LIBRARY_PATH=.
\end{lstlisting}

Weiterhin muss die in \ref{sec:app1:inst:turp} erzeugte libturpitude.so in dem Verzeichnis liegen,
aus dem Java Programmbibliotheken l"adt. Dieses Verzeichnis steht im Systemproperty \emph{java.library.path}.
Auch hier kann entweder die Bibliothek in das entsprechende Verzeichnis kopiert werden, oder 
der JVM kann beim Start ein alternativer Pfad "ubergeben werden:

\begin{lstlisting}[caption=Ausf"uhren eines Turpitude-Programmes]
# java -cp ./turpitude.jar \
  -Djava.library.path=. \
  net.xp_framework.turpitude.samples.EngineList
\end{lstlisting}

Dieser Befehl sollte nun alle verf"ugbaren ScriptEngines auflisten, Turpitude sollte hier erscheinen.


% ********** End of appendix **********
